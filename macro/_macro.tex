\usepackage{color}
%\usepackage{hyperref}

%画像と枠線をくっつける
%\fboxsep=0pt 
%\fboxrule=1pt

% 修正とTODO
\newcommand{\memo}[1]{\color{black} \color[rgb]{0.7, 0.7, 1.0} \# #1 \color{black}}
\newcommand{\todo}[1]{\color{black} \color[rgb]{1.0, 0.7, 0.7} @todo: #1 \color{black}}
\newcommand{\sodan}[1]{\color{black} \color[rgb]{0.4, 1.0, 0.1} @相談事: #1 \color{black}}

% 色付け
\newcommand{\red}[1]{\color{red} #1 \color{black}}
\newcommand{\blue}[1]{\color{blue} #1 \color{black}}
\newcommand{\green}[1]{\color{green} #1 \color{black}}
\newcommand{\cyan}[1]{\color{cyan} #1 \color{black}}
\newcommand{\magenta}[1]{\color{magenta} #1 \color{black}}
\newcommand{\yellow}[1]{\color{yellow} #1 \color{black}}

% 図の参照用
%\newcommand{\figref}[1]{図\ref{fig:#1}}
% 表の参照用
% \newcommand{\tbref}[1]{表\ref{tb:#1}}
% % 数式の参照用
% \newcommand{\eqref}[1]{式(\ref{eq:#1})}

%\renewcommand{\todo}[1]{}
%\renewcommand{\sodan}[1]{}

% イタリック 2023-12-12追記
% 以下は,2022年度のバイアス班HCI199論文において,各指標を本文に簡単にかけるようにしたもの
% わざわざ毎回\mathitなどを呼び出さないことは,記述のミス軽減と効率化につながる
\newcommand{\MT}{\texorpdfstring{$\mathit{MT}$}\xspace}
\newcommand{\ER}{\texorpdfstring{$\mathit{ER}$}\xspace}
\newcommand{\ID}{\texorpdfstring{$\mathit{ID}$}\xspace}
\newcommand{\TP}{\texorpdfstring{$\mathit{TP}$}\xspace}
\newcommand{\AIC}{\texorpdfstring{$\mathit{AIC}$}\xspace}
\newcommand{\BIC}{\texorpdfstring{$\mathit{BIC}$}\xspace}
\newcommand{\D}{\texorpdfstring{$\mathit{D}$}\xspace}
\newcommand{\W}{\texorpdfstring{$\mathit{W}$}\xspace}